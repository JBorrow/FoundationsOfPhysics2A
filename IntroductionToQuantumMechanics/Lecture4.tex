\section{Finding the time independent Schroedinger Equation}
We have the original, time-dependent Schroedinger Equation:
$$
	-\frac{\hbar^2}{2m}\frac{\partial^2 \Psi(x,t)}{\partial x^2} +
	V(x,t)\Psi(x,t) = i\hbar \frac{\partial \Psi(x,t)}{\partial t}
$$
Everything here is a function of space and time. What if $V$ is time
independent?
$$
	V(x,t) = V(x)
$$
Then the wavefunction is seprable:
$$
	\Psi(x,t) = \psi(x)T(t)
$$
This gives us:
$$
	-\frac{\hbar^2}{2m} \frac{1}{\psi(x)}
	\frac{\mathrm{d}^2\psi(x)}{\mathrm{d}x^2}
	+ V(x) = i\hbar \frac{1}{T(t)} \frac{\mathrm{d} T(t)}{\mathrm{d}t}
$$
We know, because the left-hand side is a function of only $x$, and the right
hand side is only a function of $t$, that this is equal to a constant. We'll
call that constant energy, $E$.

\subsection{RHS}
$$
	E = i \hbar \frac{1}{T(t)} \frac{\mathrm{d} T(t)}{\mathrm{d}t}
$$
Gives
$$
	T(t) = A\exp\left(-\frac{iEt}{\hbar}\right)
$$
\subsection{LHS}
$$
	E = -\frac{\hbar^2}{2m} \frac{1}{\psi(x)}
	\frac{\mathrm{d}^2\psi(x)}{\mathrm{d}x^2}  + V(x)
$$
Gives:
$$
	{\hbar^2}{2m} \frac{1}{\psi(x)}
    \frac{\mathrm{d}^2\psi(x)}{\mathrm{d}x^2} + V(x) \psi(x) = E\psi(x)
$$
This is the \emph{time independent schroedinger equation}. It also happens to be
an eigenvalue equation:
$$
	A f(x) = a f(x)
$$
Where $A$ is an operator, and $a$ is the eigenvalue. This is because we have the
Hermitian operator that operates on $\psi(x)$, giving a constant ($E$)
which is multiplied to the eigenfunction ($\psi(x)$). So we have the time-
independent schroedinger equation:
$$
	H\psi(x) = E\psi(x)
$$
Normally we get multiple energy eigenvalues, so what about general solutions? We
have that:
$$
	\Psi_n(x,t) = \psi_n(x) e^{-\frac{iE_n t}{\hbar}}
$$
Leads to the general solution:
$$
	\Psi(x,t) = \sum_n c_n \psi_n(x) e^{-\frac{iE_n t}{\hbar}}
$$
Where each $c_n$ is chosen such that:
$$
	\int^L_0 \Psi^*(x,t) \Psi(x,t) \cdot \mathrm{d}x = 1
$$

\section{Bound particles}
Let's put a particle wave in a potential. Let's have that:
$$
	\Psi(x,t) = Ae^{i(kx-\omega t)} - Ae^{-i(kx-\omega t)}
$$
Where the left-hand side is the wave travelling to the right, and the right-hand
side is the reflected wave that comes back. This reduces to:
$$
	\Psi(x,t) = 2Ae^{i\omega t}\sin kx
$$
Let's take this at $t=0$. We have that $\sin(k0) = 0$ (no surprise) and that
$\sin(kL) = 0$ because the particle is trapped in an infinite square well. This
gives us our first condition:
$$
	k = \frac{n\pi}{L}
$$
Our potential has taken a travelling wave and turned it into a standing wave(s)!
We can calculate the energy of each:
$$
	-\frac{\hbar^2}{2m} \frac{\mathrm{d}^2\psi}{\partial x^2} = E\psi
$$
Which gives us the result:
$$
	k^2 = \frac{2mE}{\hbar^2}
$$
And using the above condition:
$$
	E_n = \frac{n^2\pi^2\hbar^2}{2mL^2}
$$
We now need to normalise the wavefunction, getting:
$$
	A_n = \sqrt{\frac{2}{L}}
$$
And our energy eigenfunctions:
$$
	\psi_n(x) = \sqrt{\frac{2}{L}} \sin \frac{n\pi x}{L}
$$
Finally:
$$
	\Psi_n(x,t) = \sqrt{\frac{2}{L}} \sin \frac{n\pi x}{L}
	e^{-\frac{iE_n t}{\hbar}}
$$
Where:
$$
	E_n = \frac{n^2\pi^2\hbar^2}{2mL^2}
$$
We will probably now want to know how the probability changes as a function of
time. However, due to the $i$ in the time-dependent part - there is no time
dependence! The complex conjugate cancels out the temporal section! We can now
write down the full solution:
$$
	\Psi(x,t) = \sum_n
	\sqrt{\frac{2}{L}} \sin \frac{n\pi x}{L}
	    e^{-\frac{iE_n t}{\hbar}}
$$
Where:
$$
    E_n = \frac{n^2\pi^2\hbar^2}{2mL^2}
$$
	

\section{Wavefunctions}

By adding waves with lots of different momenta prevents us from knowing the
position and momentum at the same time. By decreasing the uncertainty in
position we increase it in momentum.
\begin{figure}[h!]
	\centering
	\includegraphics[width=0.7\textwidth]{WaveFunctionSuperpositionLecture1.pdf}
	\caption{Adding waves together ``localises" them more in space.}
\end{figure}
This gives us a weak idea of Heisenberg Uncertainty!
$$
	\Delta x \Delta p \geq \frac{\hbar}{2}
$$

\subsection{Classical Waves}

We expect classical waves to move in a certain way, in accordance with the wave
equation:
$$
	\frac{\partial^2y}{\partial x^2} =
	\frac{1}{v^2}\frac{\partial^2y}{\partial t^2}
$$
and have the following properties:
\begin{itemize}
	\item $v = \omega/k$
	\item $\omega = 2\pi f$
	\item $f = 2\pi/\lambda$
\end{itemize}
For example:
$$
	y(x,t) = A\cos(kx - \omega t) + B\sin(kx - \omega t)
$$
Which satisfies the above wave equation.

However, here, momentum ($mv$) is not dependent on wavelength. This must be
different in the quantum world, because of De Broglie's relation:
$$
	\lambda = \frac{h}{p}
$$

\subsection{Quantum Waves}

Let's just begin by studying a random wave:
$$
	\Psi(x,t) = A\cos(kx - \omega t) + B\sin(kx - \omega t)
$$
From the quantum world, we can use that:
\begin{itemize}
	\item $p = h/\lambda = \hbar k$
	\item $E = hf = \hbar\omega$
\end{itemize}
And from the classical world:
\begin{itemize}
	\item $E = p^2/2m = \hbar^2k^2/2m$
\end{itemize}
Let's begin by getting a $k^2$ out by differentiating the wavefunction with
respect to $x$ twice:
$$
	\frac{\partial^2 \Psi}{\partial x^2} = -k^2\Psi
$$
Plugging into the above formulae:
$$
	\Psi E = \Psi \hbar^2 k^2 =
	-\frac{\hbar^2}{2m} \frac{\partial^2}{\partial x^2}
$$
Now we need to get $\omega$. Let's try differentiating with respect to time.
$$
	\frac{\partial \Psi}{\partial t} =
	\omega \left[A\sin(kx - \omega t) - B\cos(kx - \omega t)\right]
$$
However, this isn't really that useful. We want it to equal a multiple of our
original wavefunction, or the maths is really hard to do.
$$
	\omega \left[A\sin(kx - \omega t) - B\cos(kx - \omega t)\right] = 
	C\Psi = C\left[A\cos(kx - \omega t) + B\sin(kx - \omega t)\right]
$$
We can now equate coefficients and find that (sub $\sin(kx-\omega t) = s$):
$$
	\omega[As - Bc] = C[Ac + Bs]
	\; \rightarrow \;
	C = \frac{\omega A}{B}
$$
And as such
$$
	B^2 = -A^2 \; \; \rightarrow \; \; B = \pm iA
$$
From this we find that:
$$
	\Psi(x,t) = A\left[\cos(kx-\omega t) + i\sin(kx - \omega t)\right] =
	Ae^{i(kx-\omega t)}
$$
And now we can put it all together to find the \emph{Schroedinger Equation for a
free particle}:
$$
	i\hbar = \frac{\partial \Psi}{\partial t} =
	-\frac{\hbar^2}{2m}\frac{\partial^2\Psi}{\partial x^2}
$$
This can be easily generalised to find \emph{The Schroedinger Equation!}:
$$
	i\hbar = \frac{\partial \Psi}{\partial t} =
	-\frac{\hbar^2}{2m}\frac{\partial^2\Psi}{\partial x^2} +
	V(x,t)\Psi
$$
We can notice that the Schroedinger Equation is \emph{linear} and
\emph{homogeneous}! This means that the linear sum of any two solutions is also
a solution.

For the classical wave equation we needed two boundary conditions to solve,
however here we only need one.

\section{Interpretation of $\Psi$}

This equation leads us to believe that there is an underlying statistical
nature to The Universe.

Classically, we know that $I \propto A^2$. Quantum mechanically we think of this
square of the wavefunction as the probability of finding the particle at that
point.
$$
	P(particle) = |\Psi(x,t)|^2 = \Psi^*\Psi
$$
We choose this because it will \emph{always} be real - like probabilities must
be.

\section{Normalisation}

Suppose we want to normalise some function $f(x,t)$. We need to find the total
area under the function and then divide through by a constant to make sure this
is unity:
$$
	|N|^2\int^\infty_{-\infty}f^*(x,t)f(x,t)\cdot\mathrm{d}x = 1
$$
And so we can find the normalisation constant, $N$:
$$
	N = \frac{1}{\sqrt{\int^\infty_{-\infty}f^*(x,t)f(x,t)\cdot\mathrm{d}x}}
$$
We simply need to do this for the wavefunction to normalise it.

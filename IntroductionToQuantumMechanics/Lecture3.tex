$$
	<E> = <T> = \frac{\hbar^2 a}{2m}
$$
We use $T$ as the operator for kinetic energy in general.

\section{The Hamiltonian}
If we take the bound particle:
$$
	E\Psi = T\Psi V\Psi
$$
And substitute in the operators:
$$
	i\hbar \frac{\partial}{\partial t} =
	-\frac{\hbar^2}{2m}\frac{\partial^2}{\partial x^2} + V = H
$$
We call this function the Hamiltonian!

\section{Relativistic systems}
If we take the relativistic energy equation:
$$
	E^2 = p^2 c^2 + m^2 c^4
$$
and plug in the operators:
$$
	- \hbar^2 \frac{\partial^2 \Psi}{\partial t^2} = 
	- \hbar^2 c^2 \frac{\partial^2 \Psi}{\partial x^2} +
	m^2 c^4 \Psi
$$
For a free particle. Now, what happens when the particle is massless..?
$$
	\frac{\partial^2 \Psi}{\partial t^2} =
	c^2 \frac{\partial^2 \Psi}{\partial x^2}
$$
And the wave equation pops out! This is really exciting - it shows that we 
with our rather crappy `derivation' of the Schroedinger Equation have satisfied
the laws of physics!

\section{Reality check}
We have a lot of $i$s floating about, so we need to check that our equations
return real answers for things like kinetic energy. Let's start with the
normalisation condition:
$$
	\intfinity \Psi^* \Psi \cdot \mathrm{d} x = 1
$$
What happens when we take the time derivative of this?
$$
	\frac{\mathrm{d}}{\mathrm{d}t} \intfinity \Psi^* \Psi \mathrm{d}x = 0
$$
We can put our time differential inside the integral:
$$
	\intfinity \frac{\partial}{\partial t} \left(\Psi^* \Psi\right) \cdot
	\mathrm{d}x = 
	\intfinity \frac{\partial \Psi^*}{\partial t} \Psi \cdot \mathrm{d}x
	+ \intfinity \Psi^* \frac{\partial \Psi}{\partial t} \cdot \mathrm{d}x
	= 0
$$
Using Hamiltonian notation to simplify:
$$
	i\hbar \frac{\partial \Psi}{\partial t} = H\Psi
$$
Which gives:
$$
	\frac{\partial \Psi}{\partial t} = -\frac{i}{\hbar} H \Psi
$$
$$
	\frac{\partial \Psi}{\partial t} = \frac{i}{\hbar} H \Psi^*
$$
Put this into the original equation to get:
$$
	\intfinity (H\Psi)^* \Psi \cdot \mathrm{d}x = 
	\intfinity \Psi^* (H\Psi)\cdot \mathrm{d}x
$$
Which proves that:
$$
	<H> = {<H>}^* \; \rightarrow \; <H> \in \mathbb{R}
$$
This means that energy is always real! We can do a similar proof that shows that
$<p> \in \mathbb{R}$. These are both Hermitian operators and as such expected
values are always real.

\section{Non-Hermitian operators}
What if we try an operator like $<xp>$? We shouldn't be able to measure these
two together to an arbritary precision!
$$
	<xp> = \intfinity \Psi x (-i \hbar) \frac{\partial \Psi}{\partial x}
	\mathrm{d}x
$$
After a bit of trivial mathematics:
$$
	<xp> = i\hbar \intfinity \Psi \frac{\partial}{\partial x} (x \Psi^*) 
	\cdot \mathrm{d}x = <px>^*
$$
Huh? What's going on? Surely this is all fine, because algebra is commutitive?
No chance. This is quantum mechanics. You don't get away with that sort of stuff
here. In fact, we have that:
$$
	px\Psi = xp\Psi - i\hbar \Psi
$$
Meaning that:
$$
	(px - xp)\Psi = -i\hbar\Psi
$$
The order of operations matters. We define this as a commutator, with the
notation:
$$
	[x,p] = xp-px = i\hbar
$$
and this one is called the fundamental commutator. We say that if:
$$
	[A,B] = AB - BA \neq 0
$$
the commutator is non-hermitian and as such not real.
\paragraph{Example} with the Gaussian!
We find that:
$$
	<xp> = \frac{1}{2}i\hbar, \; \; <px> = -\frac{1}{2}i\hbar
$$
which satisfies
$$
	[x,p] = xp-px = i\hbar
$$
Even though these are non-reals, we can take the mean to find some sort of a
value (as we know $<x> = 0$, $<p> = 0$ and as such expect $<px> = 0$):
$$
	\frac{1}{2}(xp+px) = 0
$$

\section{Commutators}
If you have a commuator $[A,B] \neq 0$, then we say that it \emph{doesn't
commute}. In physical terms this means that you've hit the uncertainty principle
and are trying to measure $p$ and $x$ simultaneously.

\subsection{Commutators and the uncertainty principle}
We can have a formal treatment of this, and it is reasonably short:
$$
	\sigma_A^2 \sigma_B^2 \geq (\frac{1}{2}<[A,B]>)^2
$$
Let's try with our $[x,p]$ commutator:
$$
	\sigma_x^2 \sigma_p^2 \geq \left(\frac{i\hbar}{2i}\right)^2 = 
	\left(\frac{\hbar}{2}\right)^2
$$
Meaning:
$$
	\sigma_x \sigma_p \geq \frac{\hbar}{2}
$$
We chose the Gaussian function because it is the wavefunction that minimises
the uncertainty, and for the Gaussian:
$$
	\sigma_x \sigma_p = \frac{\hbar}{2}
$$

